\documentclass{beamer}
\usetheme{Boadilla}
\usepackage{outlines}

\title{Git Basics}
\author{Steve Offutt}
\institute{OSSEM}
\date{\today}


\begin{document}
\begin{frame}
    \titlepage
\end{frame}

\begin{frame}
    \frametitle{Outline}
    \tableofcontents
\end{frame} 

\section{What is version control?}
\begin{frame}
    \frametitle{What is version control?}
    \begin{itemize}
        \item{From Git's Website:}
            "Version control is a system that records changes to a file or set of files over time so that you can recall specific versions later."
        \item{Keep track of stuff! Software projects, documentation, anything really.}
    \end{itemize}
\end{frame} 

\section{Installing Git}
\subsection{Windows}
\subsection{Linux}
\subsection{OSX}
\begin{frame}
    \frametitle{Installing Git}
    \begin{itemize}

        \item{Windows}
        \begin{itemize}
            \item{Got to \url{https://git-scm.com/download/win} and download the installer.}
            \item{This will install the Git shell and setup the windows environment for Git.}
        \end{itemize}

        \item{Linux}
        \begin{itemize}
                \item{Should be installed by default. If not, use your distributions package manager to install Git from the repositories.}
                \item{Example: Ubuntu, type sudo apt-get install git}
        \end{itemize}

        \item{OSX}
        \begin{itemize}
            \item{I don't know. Google it.}
        \end{itemize}

    \end{itemize}
\end{frame} 

\section{Basic Commands}
\begin{frame}
    \frametitle{Basic Commands}
    \begin{itemize}

        \item{init}
        \begin{itemize}
            \item{This is what you would use to typically start a NEW git repository.}
            \item{Suppose you wanted to start a project called CoolProject in /home/OSSEM/Projects/}
            \item{You would use the command git init CoolProject}
            \item{Just like that the CoolProject directory is made along with all of the repository files/folders.}
        \end{itemize}

        \item{clone}
        \begin{itemize}
            \item{Used to clone a repository.}
            \item{Lets clone the DC Darknet Github defcon24 repo with git clone https://github.com/thedarknet/defcon24.git}
            \item{The ENTIRE contents, including branches and commit histories, will be cloned to your working directory. How cool!}
        \end{itemize}
        
        \item{status}

        \item{add}

        \item{commit}

        \item{branch}

        \item{push}

    \end{itemize}
\end{frame}

\end{document}
