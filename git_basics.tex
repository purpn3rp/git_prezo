\documentclass{beamer}
\usetheme{simple}
\usetheme{Boadilla}
\usepackage{lmodern}
\usepackage[scale=2]{ccicons}
\usepackage{hyperref}

\setwatermark{\includegraphics[height=8cm]{img/logomark-orange-color.png}}

\title{Git Basics}
\subtitle{Taken from a complete n00b}
\author{Steve Offutt \& Joshua Loscar}
\institute{OSSEM}
\date{\today}


\begin{document}
\begin{frame}
    \titlepage
\end{frame}

\begin{frame}
    \frametitle{Agenda}
    \tableofcontents
\end{frame} 

\section{What is git?}
\begin{frame}
    \frametitle{What is git?}
    \begin{itemize}
        \item{From Git's Website:}
            "Version control is a system that records changes to a file or set of files over time so that you can recall specific versions later."
        \item{Keep track of stuff! Software projects, documentation, anything really.}
        \item{git is decentralized version control. The repository is on your local machine, always.}
            \begin{itemize}
                \item{This means that if you are on the road and need to roll back to an older version, no problem! Just type the commands.}
            \end{itemize}
        \item{"Ry's Git Tutorial" free kindle book. I read half of it. It's amazing for \$0.}
    \end{itemize}
\end{frame} 

\section{Installing Git}
\subsection{Windows}
\subsection{Linux}
\subsection{OSX}
\begin{frame}
    \frametitle{Installing Git}
    \begin{itemize}

        \item{Take it away Josh!}
        \item{\url{https://docs.google.com/presentation/d/1BOHQ-I8VuJIatDCqHs3OnyuDk2UghcoFg9Bbr7ircwk/edit?ts=57d2eb57\#slide=id.p}}
        %\item{Windows}
        %\begin{itemize}
        %    \item{Ask Josh ;-)}
        %    %\item{Got to \url{https://git-scm.com/download/win} and download the installer.}
        %    %\item{This will install the Git shell and setup the windows environment for Git.}
        %\end{itemize}

        %\item{Linux}
        %\begin{itemize}
        %    \item{Ask Josh ;-)}
        %    %    \item{Should be installed by default. If not, use your distributions package manager to install Git from the repositories.}
        %    %    \item{Example: Ubuntu, type sudo apt-get install git}
        %\end{itemize}

        %\item{OSX}
        %\begin{itemize}
        %    \item{Ask Josh ;-)}
        %    %\item{I don't know. Google it.}
        %\end{itemize}
    \end{itemize}

\end{frame} 

\section{GUI vs CLI}
\begin{frame}
    \frametitle{How to use git? GUI or CLI? Choose wisely...}
    \begin{itemize}
        \item{The most universal way of using git is via the command line}
        \item{A lot of git GUIs are very nice. A visual really helps with branches.}
        \item{GitKraken is pretty cool looking! Cross platform and free (I think?).}
        \item{giggle and gitg both look pretty nifty as well.}
        \item{There are tons. To get you started check out \url{https://git-scm.com/downloads/guis}}
        \item{What makes sense for your project?}
    \end{itemize}
\end{frame}

\section{Staging?}
\begin{frame}
    \frametitle{WTF does staging in git mean?}
    \begin{itemize}
        \item{In git you must "stage" or "unstage" items prior to commits.}
        \item{Think of this as "setting the stage" for your commit.}
            \begin{itemize}
                \item{All my stuff is set, now I'm ready to do my thing!}
            \end{itemize}
        \item{This is very important when using git as to make sure everything you do or don't want added to your project is or is not added.}
    \end{itemize}
\end{frame}

\section{Basic Commands}
\subsection{init, clone, status, add}
\begin{frame}
    \frametitle{Basic Commands}
    \begin{itemize}

        \item{\texttt{git init}}
        \begin{itemize}
            \item{This is what you would use to typically start a NEW git repository.}
            \item{Suppose you wanted to start a project called CoolProject in /home/OSSEM/Projects/}
            \item{You would use the command \texttt{git init CoolProject}}
            \item{Just like that the CoolProject directory is made along with all of the repository files/folders.}
        \end{itemize}

    \item{\texttt{git clone}}
        \begin{itemize}
            \item{Used to clone a repository.}
            \item{Lets clone the DC Darknet Github defcon24 repo with git clone https://github.com/thedarknet/defcon24.git}
            \item{The ENTIRE contents, including branches and commit histories, will be cloned to your working directory. How cool!}
        \end{itemize}
        
    \item{\texttt{git status}}
        \begin{itemize}
            \item{Display the state of the current branch.}
            \item{Also displays the current branch. Just in case you forgot.}
            \item{Files that have been added or deleted from the repo.}
        \end{itemize}

    \item{\texttt{git add}}
        \begin{itemize}
            \item{Simply stage files or directories.}
        \end{itemize}

    \end{itemize}
\end{frame}

\subsection{commit}
\begin{frame}
    \frametitle{\texttt{git commit}}
    \begin{itemize}
        \item{You must stage your items in your repo prior to commiting.}
        \item{All unstaged items will not be brought into the commit.}
        \item{This command will take your current staged items and commit your changes!}
        \item{Git forces you to say something about the commit.}
        \item{If only the command git commit is typed, git will launch the editor of your choice defined by the \texttt{\$EDITOR} environment variable}
        \item{After entering your message into the editor, git will continue with the commit.}
        \item{It is also recommmended that you run \texttt{git status} prior to commiting to ensure that everything is there.}
        \item{Your name and email typically should be set before submitting a commit.}
    \end{itemize}
\end{frame}

\subsection{branch}
\begin{frame}
    \frametitle{What is a branch?}
    \begin{itemize}
        \item{A branch is an independent line of development.}
        \item{Say you want to try this one "thing" out without touching the master branch.}
            \begin{itemize}
                \item{Cool story, bro. Make a branch!}
            \end{itemize}
        \item{\texttt{git branch branch\_name} will make a branch called branch\_name.}
    \end{itemize}
\end{frame}

\subsection{merge and push}
\begin{frame}
    \frametitle{\texttt{git merge} and \texttt{git push}}
    \begin{itemize}
        \item{\texttt{git merge}}
        \begin{itemize}
            \item{This command merges your changes on a branch to another branch.}
            \item{If there are conflicts, git will tell you about the conflicts and also report in the file exactly where the conflict is. Helpful if it is source code.}
            \item{Executing \texttt{git merge branchA} from the master branch will merge changes from branchA into master.}
            \item{This also works for merges between branches(I think).}
        \end{itemize}

    \item{\texttt{git push}}
        \begin{itemize}
            \item{This command takes your commits and pushes them to the remote repository.}
            \item{Of course you must have permission to submit changes to the result repo...}
            \item{You config file must be set up prior to pushing!}
        \end{itemize}

    \end{itemize}
\end{frame}

\section{Who cares?}
\begin{frame}
    \frametitle{Who cares?}
    \begin{itemize}
        \item{Without a doubt people in this room who deal with software care.}
        \item{What about OSSEM? Could we use git for...}
            \begin{itemize}
                \item{Badge reader}
                \item{Presentation templates}
                \item{By-laws. Do we even have those?}
                \item{Keeping track of projects}
            \end{itemize}
        \item{OSSEM Github started 9/9/2016! w00t!}
            \begin{itemize}
                \item{\url{https://github.com/OSSEM-Holon}}
            \end{itemize}
    \end{itemize}
\end{frame}

\section{References}
\begin{frame}
    \frametitle{References. Take a look, it's in a book...}
    \begin{itemize}
        \item{Git's website. They have a (free)book!}
            \begin{itemize}
                \item{\url{https://git-scm.com/book/en/v2}}
            \end{itemize}
        \item{"Ry's git Tutorial" by Ryan Hudson on Amazon.com}
            \begin{itemize}
                \item{\url{https://www.amazon.com/Rys-Git-Tutorial-Ryan-Hodson-ebook/dp/B00QFIA5OC/}} 
            \end{itemize}
    \end{itemize}
\end{frame}
\end{document}
